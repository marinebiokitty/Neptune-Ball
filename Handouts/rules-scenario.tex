\documentclass[sheet]{NeptuneBall}

%% document-wide tweaks
\interlinepenalty10000
\setstretch{1}
\def\mytype{Rules and Scenario}
\lfoot{}\rfoot{}
\parindent0pt

\begin{document}

%% layout for cover page
\thispagestyle{empty}
\parskip0pt

%% title box
\begin{center}\LARGE\bf\begin{tabular}{|c|}
  \hline \gamename\\ \gamedate\\ Rules and Scenario\\ \hline
\end{tabular}\end{center}

\vfill\vfill

%% player side of the GM/player contract
The following are the rules for {\em\gamename}, a real-time, real-space roleplaying game sponsored by the Luminary Roleplay Society, and running at Intercon U, 2023. You are responsible for knowing and following these rules.  Many of them are nigh-impossible to enforce and rely upon the honor system.

\vfill

%% GM side of the GM/player contract
The {\bf gamemasters} ({\bf GMs}) run the game.  If you have any problems or questions concerning the game, contact a GM.  Rulings they make are final.  They may violate the letter of the rules to preserve the spirit.  The GMs promise to be as fair and reasonable as possible. Neither they nor these rules are perfect.

\vfill

%% have fun
This game is intended to be fun.  Getting into character, roleplaying, being dramatic, and playing competitively can all increase the fun of the game.  Do not take the game too seriously.  Even if you are losing, keep a good attitude.  When the game is over, the real winners are the players with the best stories.

\vfill

%% be safe
This is only a game.  Everyone involved should act with courtesy, sportsmanship, patience, and taste.  The GMs may expel anyone they believe to be violating the spirit of the rules or the game.  Emotions may run high.  If you think things are crossing the line from game to reality too much, or if you are just getting too stressed, do what you need to in order to help yourself calm down including maybe taking a break.  Stay in control.  Use common sense.  Always, play safely, then play to have fun.

\vfill

%% disclaimer and copyright
%% author list auto-generated from Lists/gm-LIST.tex
This game is a work of fiction.  Although it may refer to things in the real world, it does so only for the sake of the scenario.  It does not represent the opinions of the GMs or the Luminary Roleplay Society. These rules are modifications of those used in previous games.  This game and all materials thereof are copyright 2014 by Acata Felton, Jeremy Cole and the Stanford Gaming Society.

\vfill\vfill

\begin{center}\bf
  BROUGHT TO YOU BY THE LUMINARY ROLEPLAY SOCIETY
\end{center}

\vfill

\clearpage

%% layout for Table of Contents page
\thispagestyle{empty}
\tableofcontents

\clearpage

%% layout for main body of rules
\setcounter{page}{1}
\parskip5pt


%% The Scenario should present the setting of the game (including time
%% and place).  It may include basic history and culture.
%% Sufficiently long and/or complicated games might have a full
%% timeline.
%%
%% You may also want to give meta-information like basic roleplaying
%% and costuming hints.

\section{Scenario}

\cKing{\King} \cKing{} is delighted that you have agreed to attend the Neptune Ball. This annual bash is the most exclusive, and grandest ball in all of \pAtlantis{} to celebrate the turning of the year. There will be dancing, refreshments, and a grand banquet, with all of the finery the palace has to offer.

In attendance will be many of \pAtlantis{}'s social and political elite. For the first time in a hundred years of war, the Neptune Ball will also be host to talks of peace with \pPacifica{}. Witness history being made - or have your voice heard, and help steer your country in the right direction. Should politics bore you, there will always be the intrigue of court to divert you. Be careful though, in pursuit of other's secrets, you may find your own revealed as well.

\emph{Unless you know otherwise, your \textbf{character} is free to leave gamespace by going up the main door to the room and declaring your intentions with a 15 count, but your character MAY NOT return if you choose to leave. {\bf Players} may step out at any time for self-care.}

\subsection{Costuming:}
Costuming is always admired but never required. Please prioritize your comfort, safety, accessibility, time, and financial limitations. If you want to show up in a full ball gown go for it. If you want to show up in jeans and a tee-shirt, go for it. If you want to show up in the costume you're gonna wear to the next game, go for it.

That said, some of you may enjoy some costume suggestions:
\begin{enumerate}
	\item The premise of the game is an extremely exclusive new years eve party thrown by the ruler of an entire nation. Cocktail attire or fancier would not be out of place.
	\item Unless your character blurb told you otherwise, you are a mer-person. Now, most of us don't have full merfolk tails lying around, and even if we did, wearing one would make it difficult or impossible to move around. If you wish to evoke such an aesthetic, you can:
	\begin{enumerate}
		\item Use fabrics that evoke scale patterns.
		\item Use beading and/or sequins to create sparkle.
		\item Emphasize blues and greens in your outfit.
		\item Include accessories like sea shells, starfish, netting, etc. 
		\item If you have the option, a trumpet style skirt can create a pseudo fish-tail silhouette. Alternately long flowy clothes also work well to evoke the idea of floating in the water.
	\end{enumerate}
	\item If your character is obviously based on a character from the Disney movies, feel free to use them as inspiration (or not).
	\item If your character is royalty, we will have crowns available for your use during the game. You are of course welcome to supply your own if you prefer, or decline to wear one at all.
\end{enumerate}

\paragraph{Game Times:} Game runs from 9am to 1pm on {\bf Saturday, March 4th} in the Tiverton Room. Surviving PCs are expected to be in-game for the entirety. {\bf Please be on time} as game runs 3.0 hours exactly and we need time to get set up together beforehand (i.e.: We need to give you hard copies of all of your game material).  We also have a tight turn around in the room for the next game, so being able to start on time is important for making sure we have time for post game deroll and debrief within our time slot.

\section{Safety:} This is a game.  Real violence is unacceptable. Game action should cause no real-world damage, either to people or property.  If something dangerous is happening, call a halt.  Stay in control, use common sense, and do not endanger yourself or others. You should not run or otherwise force your way into or through someone else's ZoC, and you should not make physical contact with another player without permission.

This game employs the following safety mechanics: ``Open Door,'' ``Cut and Brake,'' and ``Okay Check-In.'' The OOC symbol is also always available for use as a safety mechanic.

\subsection{Potentially Activating Topics:} As described in the game advertisement, and further detailed in the casting survey, this game deals with some very heavy topics. We have done our best to respect what folks shared with us when we did casting, but there is always a possibility that we missed something, or that something unexpectedly activates you. We have safety mechanics for exactly this reason. They are tools to help you take care of yourself physically and emotionally, as well as help you take care of your community. Your GMs are also here to help support you. 

\subsection{Taking the Heavy Stuff Seriously:} Many of the topics addressed in this game have real world analogs. They are intentionally included as part of this game as a chance for players to consider the impact they can have on individuals, groups, countries, and policy. While playing this game, some of the weight can get lost in the adrenaline and excitement of the moment. The GMs encourage you to sit with some of the subjects before and after the game, and reflect on how you can treat the heavy topics with respect, and whether you learned anything new or interesting from having engaged with it during play. Immersive experiences like LARP can be powerful tools for building new levels of understanding around the experiences of fellow humans.

\section{Getting Started}

%% Character packets come first, since they are the tangible things
%% handed to players.  Also a convenient place to define Player
%% Character.
\subsection{Character Packets}

Your character packet is a big manila envelope.  It contains your
role: who you are, what you're up to; everything about your part as a
{\bf player-character} ({\bf PC}) in the game.  Read all the contents
and generally keep them with you during the game.  If you are missing
something or find something which doesn't seem to belong to you, tell
one of the GMs.  Character packets are confidential.  Game materials
which cannot be given to other players are marked ``Not
Transferable,'' whereas things which can be given to others are marked
``Freely Transferable'' or ``Game Item.''

Your Character Packet would normally contain:
%% other things your game uses, like money, should be described below

\paragraph{Name-Badge:} A name-badge with your Character name and pronouns, character
description, and {\bf badge number} on it shows that you are in the
game; wear it visibly while you are playing.  It represents your
character's body in-game.  Badge numbers are not in-game information.
See the {\em Character Bodies} and {\em Badge Numbers} sections for
more details.

\paragraph{Character Sheet:} Your character sheet describes who you
are and what you are up to.  It contains a list of everything else
that should be in your character packet.  Do not show or read your
character sheet to other players.

\paragraph{Bluesheets:} A bluesheet describes information common to
members of a group.  When in conflict, character sheet information
overrides bluesheet information.  Do not show or read a bluesheet to
other players.

\paragraph{Greensheets:} A greensheet describes and expands abilities,
mechanics, or in-game knowledge.  Do not show or read a greensheet to
other players.

\paragraph{Stat Card:} Your stat card lists your statistics.  \textbf{You
might not know what all of your stats mean. This is intentional}  Do not show your stat card to
others. Do not share individual stat values unless prompted by a mechanic.  The reverse side is a {\bf death report}; fill it out and
give it to the GMs when your character dies.

\paragraph{Ability Cards:} An ability card explains a special ability
your character has.  The front side describes the effects; show it to
players when you use the ability (the "ability effect" side).  The reverse is the rules of use and
must not be shown to other players.

\paragraph{Memory/Event Packets:} A memory packet is an envelope or
stapled piece of paper with a {\bf trigger} which describes when to
open and read it.  If the trigger is a number, open the packet when
you see something with that number.  If it's a quoted phrase, open
when you hear or read it in-game.  If it's a symbol, open when
instructed.  Do not take game action based on an unopened trigger.  Do
not show or read a memory packet to other players.

\paragraph{Research Notebooks & Memory/Event Books:} A notebook/book is a series of pages stapled together into something that resembles a little booklet. Each page will be folded over and stapled shut. For \textbf{Research Notebooks}, there are no triggers to open the pages, instead they are numbered "page 1," "page 2," etc. \textbf{Open page 1 as soon as you get your character packet! This will be just before game start.) Each page of the notebook will give you instructions on how to open other page(s) to progress through the content and discover or accomplish something cool IC. For \textbf{Memory Books}, each page has a unique trigger printed on it. When you meet the requirement, open the page to learn what happens. \textbf{Some characters may have notebooks that work differently. Ask a GM if you aren't sure how to interact with your notbook.}

\paragraph{Items:} In-game items may be transferred from character to
character, and should be marked as such.  See the {\em Items Etc.}
section for more details.

\paragraph{Electronic Copies:} You should have receieved electronic copies of your character sheet, and any bluesheets, green sheets, white sheets, and abilities your character has. Not all characters have all of these types of documents. Your character sheet will list them at the bottom. You will \textbf{not} recieve electronic copies of your stat card, items, any memory packets, or any notebooks. These will be availble as hard copies along with the rest of your packet at the vent.

%% Some Assassin Game fundamentals
\subsection{Reality and Game Reality}

There is a big difference between reality and game reality.  Players must treat each other with courtesy and explain to each other what their characters perceive in confusing situations; e.g.\ ``My character's hands are covered in blood,'' an {\bf out-of-game} statement.  Characters are under no such restrictions, and may do what it takes to further their goals; e.g.\ ``Uh, hi Bob.  Just got back from the butcher shop,'' an {\bf in-game} statement.

{\bf Metagaming} is inferring in-game knowledge that is inappropriate for your character from out-of-game information.  Do your best to not metagame and especially to prevent the risk of metagaming.  Be your own harshest critic.

\paragraph{Halts:} A halt pauses game action.  To call one, say ``game halt'' in a clear and audible voice; players in half of the room should be able to hear you; but you don't need to shout. End a halt by saying ``three, two, one, resume.''  You may need to call a halt because: a rule instructs you to, for safety and similar out-of-game issues, or to pause game and fetch a GM for any reason (you can send an NPC if one is around).

\paragraph{Not-Here:} You may go not-here by turning your name-badge around so the ``I'm Not Here'' side is showing (or by removing your badge entirely, if you are leaving game).  Putting a hand on your head, visible from a distance, helps if you're near other players.  YOu may ned to go ``Not Here'' because: a rule instructs you to, you need to leave game, or to fetch a GM while in a halt.

%% last two sentences not for closed-time/space game
When you are not-here, your character is not there.  Your character cannot see, hear, or remember any game actions or information you (the player) happen to encounter.  Avoid other characters, common game areas, game signs, or any sort of game interaction.  

Please help us avoid scaring people unnecesarily. Unless it is true in real life, please do \textbf{not} shout ``fire'' or ``gun,'' or any such similar words (we don't expect these words to come up but this is just in case). These words are used in real life to indicate very serious dangers, so it is understandable that people have a big reaction when they hear these words. If you must use these words, or words similar to them in game, please try to do so in a stage whisper rather than yelling at the top of your lungs.

\paragraph{Mechanics:} Many actions your character can take, such as
walking, talking, and general interaction with other characters, are
represented by you doing them.  Others, like combat, are performed via
abstract mechanics, which are described in ability cards, greensheets,
and rules.  The abstract information for mechanics (like badge
numbers) may not be discussed in-game.  If you want to do something
special for which there is no mechanic, ask a GM.

Become familiar with your mechanics before game starts, especially
those which occur under time-pressure (like combat).  Game action will
not stop for memory packets, greensheets, or such.

A {\bf kludge} (and derivative forms like ``kludge-ite'') is something
impervious to logic and cleverness, usually for game-balance.  You
can't affect a kludge without a specified mechanic.

{\bf Zone of Control} ({\bf ZoC}) is a rough distance measurement.
You are within ZoC of someone if your outstretched fingers can touch
their outstretched fingers.  Double-ZoC is twice this distance,
triple-ZoC is three times, etc.

{\bf Headbands} represent obvious visual effects; wear them visibly on
your head.  If you see a headband and don't know what it represents,
ask.  If you are wearing a headband, tell people what their characters
see. See the end of this document for additional details

An {\bf interruptible} mechanic has some duration, and may involve
continuous roleplaying.  It is stopped if you are attacked or if
someone within ZoC says {\bf ``I stop you''} or an equivalent phrase.
Some mechanics may be easier or harder to interrupt.

A {\bf n-count} is an interruptible mechanic with a repeated, counted
incant (``I pour a drink one, I pour a drink two, I pour a drink
three'').  Speak clearly; each count must take at least a full second.
Each n-count will specify the number, e.g.\ a 3-count.

\subsection{Basic Strategy}

Make sure you understand the rules.  If you are completely confused,
get a GM who will try to help you out.  Make sure you know enough
about your character to role-play him or her when you start talking to
other people.  Read through your entire packet a couple of times, and
skim through it again right before game starts.  If you don't know
something about your character, ask a GM.

As a character, your first priority should be to open lines of
communication.  Contact people, show up at meetings, and chat.  Try to
be easy to get in touch with.  Ask people questions on relevant
subjects.  They'll probably lie, but you may find something out.

There are no guarantees that you can trust anyone, but since
cooperation is the key to accomplishing things, you will be forced to
trust people anyway.  The most trustworthy people are probably those
who need you.

\textbf{Those who do not study history are doomed to repeat it. This game has a lot of history in it. You should strive to learn as much as you can about your history, and the history of those around you.}

\section{Items Etc.}

Many in-game items are represented by little white cards with a number
and description.  Item cards may be shown to others, passed around,
stolen, etc.  The {\bf item number} on the card is not in-game
information and may not be discussed.  

Some mechanics in game may involve making item cards. Such items should be clearly marked as ``in game'' and treated as such.

Use common sense.  You can't carry a hundred rocks in your pocket,
fold a sword in half, or hide a life-sized statue in a fire hose.  You
can't stop a bullet with a set of blueprints or rip apart a metal safe
with your bare hands.  Even if your bag can carry a shovel in it, the
shovel noticeably sticks out (``you see a shovel sticking out of my
bag'').

\paragraph{Written Information:} If you write in-game information down
on a piece of paper, that paper is now an in-game item and must be
clearly marked as such.  Don't write in-game information on
out-of-game documents (character sheet, etc.).  Don't write
out-of-game information (like memory packet triggers) on in-game
documents.

\paragraph{Envelopes:} Some items and locations may have an attached
envelope (or just be a labeled packet or folded paper).  The envelope
may include directions for when to open these (``open packet if you
press the big red button'' or ``open packet if you eat this'');
otherwise you may only open them if instructed.  Close them when you
are done.  Open and close packets gently.

\paragraph{Signs:} Some locations and other game materials are
represented by signs or packets posted throughout game area.  You may
read any signs and must follow any rules printed on them.  If a sign
or packet doesn't have some sort of in-game description (it only has
out-of-game mechanics information, like a number or just a colored
dot), then your character doesn't even see it or know that anything
unusual is there.

\paragraph{Bulkiness:} A bulky item is too big or heavy to be carried
or concealed freely.  Bulkiness is measured in {\bf hands} or {\bf
dots} (how many hands it takes to carry it).  If you are carrying a
bulky item, make it clear to onlookers (hold the card).  A hand
carrying a bulky object may do nothing else.  With one hand less than
required, you may drag a bulky item at a slow pace.

\paragraph{Valuable:} Some items are marked ``valuable''. Some plots may require you to acquire valuable items. Any item that has this tag qualifies.

\paragraph{Props:} Some items may have props (physical representations
or {\bf physreps}) associated with them.  The card and physrep should
be kept together.  If they are separated, the card is the real item.
Prop items are as bulky as the physrep.  They can be carried in bags
that can hold them, on straps that are attached to them, etc.

\paragraph{Character Bodies:} A body is {\bf three hands bulky} and
usually represented by a name-badge.  It must be willing or unable to
resist for you to carry it.  Carry the badge conspicuously.  Onlookers
can't tell if it's dead without close examination, unless it would be
obvious (like headless).

\paragraph{Unstashable Items:} Unstashable items can't be hidden or left behind.  They look too important, valuable, or interesting; NPCs will not let them stay there.  These include any item that has a physrep. This is a kludge.  If you're not leaving an unstashable item in another PC's care, and you want to leave it behind, give it to an NPC, or GM.  You may leave it in plain sight in a public area if there are other PCs around.

\subsection{Searching, Stashing, and Stealing}

\paragraph{Places:} To search a place, search it.  Normal items can be stashed in any reasonable, legal place.  Don't put items in places where retrieval might be dangerous; consequently, don't go rummaging through such places for game items. Don't stash or search in places that are not in-game. Aka: Keep all game materials inside the room where we are playing.

\paragraph{People:} All searches of characters or their belongings are conducted via player dialogue.  Someone must be willing or unable to resist for you to search them.  You need at least one free hand to search someone.  Searching is interruptible (see above).

You can perform a {\bf pat-down search}, which will only reveal the presence of weapons.  This takes as much time as it takes your victim to tell you what you find.  If you're the victim, do this at a reasonable pace.

A {\bf total search} is an invasive, complete search of a character's clothing.  This reveals all in-game items, and takes as long as your victim spends handing over possessions.  If you're the victim, hand over items at a reasonable pace. {\bf Items labeled ``magical effect'' are never revealed during searches unless you know otherwise.}

\paragraph{Bags:} To search a bag in someone's possession, say ``I search your bag.''  This proceeds just as a total search.  To search an unattended bag, search the physrep.  Don't look through someone's character packet, read their journal, steal their dinner, etc.  If the bag has an attached, displayed item card with an envelope, the bag is a prop; search the envelope and not the bag.

If you want to leave in-game items in an unattended bag (e.g.\ to hide a bomb), keep items in reasonable places that could be found with a quick search of the bag.  Don't hide in-game materials mixed together with out-of-game materials.  You can attach an item card and envelope to segregate in-game items from out-of-game materials.


\section{Violence, Damage, and Death}

\subsection{Health States}

Characters have five possible states, concerning health and damage. When you are {\bf fine}, you may act freely.  When you are {\bf restrained}, you are helpless and may do nothing but talk.  When you are {\bf knocked out}, you will wake up in five minutes.  When you are {\bf wounded}, you are unconscious, bleeding, and will die in five minutes.  When {\bf dead}, you are dead.

When knocked out or wounded, fall down and drop anything you are holding.  Just lie there.  You won't be doing much of anything until you wake up.  Do not listen to conversations going on.

Dead men tell no tales.  If dead, do not give out any information about your character or death to any players.  You may remain on the scene to play the part of your corpse; describe obvious information to onlookers (``I have a gunshot wound in my back''). When you leave, place the front of your name-badge with a description of the body's obvious state.  Take the ``I'm Not Here'' side to wear. Stack your items with your body. Make sure the GMsknow about your death.  If your death becomes generally known to the other characters, you may be able to become an observer.  Until the game is over, you may not convey game information to any player.

\subsection{Weapons}

Weapons are represented only by item cards in this game (no phys reps). Weapon effects are on the card.  To use a weapon, you must have it in your hand and unobstructed.  Display it in an obvious manner.  You cannot hold more than one weapon in a hand.  You may only use one melee weapon at a time.

\subsection{Killing Blow}

Unless you know otherwise, killing blows are forbidden in this game. A character may still bleed out and die.

\subsection{Martial Combat}

%% intro
All characters have a {\bf Combat Rating} ({\bf CR}) stat.  This represents your basic skill in martial combat; you use the same number for attacking and defending.  Someone with a CR of one can't fight very well.  Someone with a CR of three is somewhat burly or skilled. When using this stat, you may pull your punches by using a lower number.

%% offense
To martial-attack someone, clearly state your attack and CR (``\aKnockOut{} 2'', ``\aWound{} 2'', etc.) from within ZoC.  You need the ability card for any attack you make; you don't have to display it.  Your attack must resolve before you make another; otherwise, you may act freely.  If an ally directs {\bf \aAssist{}} at you after you attack, you may, within 2 seconds, restate your attack with the \aAssist{}'s CR added (``\aWound{} 3'', ``\aAssist{} 2'', ``\aWound{} 5'').  \aAssist{} does not change your CR for defense.  You may ignore an \aAssist{}.

{\bf If you wound someone or assisting in a wound attack,} you must put on a red headband for 10 minutes. This represents the fact that it is obvious that you are bloody.

Wound attacks require a \iKnife{}. Restrain attacks require rope (which is freely available). Knock Out attacks do not require any item.

%% defense
When martial-attacked, resolve by comparing the attack against your CR.  If your CR is lower, take the effects; else, say ``{\bf resist}'' and the attack has no effect.  If you neither say ``resist'' nor state your own attack within two seconds of the incant's end, you are surprised and the attack just works.  The attack begins when the incant begins; until you resolve, all of your actions other than martial attacks are interrupted; serial attacks don't prevent simple
actions (talking, weapon-drawing, ranged attacks) in-between.  Resolve all attacks alone, in the order they occur; choose the order if it is unclear.  If you are attacked with ``{\bf waylay}'' instead of a CR (``\aKnockOut{} waylay''), the attack just works.

\paragraph{Martial Attack Abilities:} Here is a list of attack abilities.  You should assume that every character has \aKnockOut{}, \aWound{},  \aAssist{}, and \aRestrain{}.  Other attack abilities may exist.\nopagebreak

\begingroup
  %% complicated typesetting
  \MAP{Abil}{%
    \setbox0\hbox{\phantom{w}{\em Effect}: \MYeffect}%
    \par{\bf\MYname}: \MYtext\hfill\null\hskip\wd0\null%
    \hskip-\wd0 plus1fill\box0%
    \nopagebreak\par%
    }
  \aKnockOut{}
  \aWound{}
  \aAssist{}
  \aDisarm{}
  \aRestrain{}
\endgroup

\subsection{Stealth}

Stealth abilities represent sneaking up on a victim with obvious intent to invade their personal space, probably to attack them by surprise or to pick their pocket.

To use a stealth ability, you must be within ZoC of your victim.  Form a llama with your hand (index and pinky fingers extended, thumb touching the other two fingers extended ) and extend it along the direct, unobstructed line from your shoulder to the victim's head.  Hold this position for the time specified by your ability.  Before this time is up, the ability is thwarted if anyone attacks you or if the victim notices the symbol.  If they react in any way to the symbol, they have noticed; you (the attacker) make the call.

If you notice someone using a stealth ability on you, make it obvious. ``I notice you'' is unambiguous; use it if you can.  Once a stealth ability is finished, you may not retroactively have noticed.

\paragraph{Waylay:} You can attack by surprise as a stealth ability. You must hold the symbol for five seconds.  If you succeed, you may replace your CR with ``waylay'' for a single immediate attack on your victim.


\paragraph{Rope:} Rope is freely available as seaweed.  Make an item card for it. To tie someone up, they must be either willing or helpless.  If you
get tied up with rope, you become restrained.  If you are conscious and left alone, you can wriggle free in five minutes.  Rope is a one-hand bulky item.

\paragraph{Doors and Locks:} Some doors or items in game are {\em locked}. You may not open them or get past them unless you fit the requirements listed, or have some other method of opening locks. Closing such an item or door locks it again.

\section{Miscellaneous}

\paragraph{Headband Colors:} Differently colored head bands are used
in this game to represent obvious aspects of a players appearance. They are
also used to delinate GMs and observers.
\begin{enumerate}
  \item A white headband represents GMs and observers.
  \item A green headband represents a human.
	\item A blue headband represents an NPC.
  \item A red headband represents a bloody character. This is immediately visible to all observers. Red headbands are acquired by performing wound attacks or assisting in such attacks.
  \item A yellow headband represents a shark. Don't get too close!
  \item A black headband represents a terror of the deep. You should ask what you see from a safe distance.
  \item If you see another color head band, you should ask the player what you see.
\end{enumerate}

Only humans and merfolk are publically known to be attending the Neptune Ball.

\paragraph{Badge Numbers:} The first digit of your badge number is your character's apparent age in decades. All bodies are 3 hands bulky.

\section{The Neptune Ball Specific Changes}
This section is a recap of the changes specific to this game from other
games that use a similar rules set.
There are also several new and important game-wide mechanics. Familiarize
yourself with them before game.

\subsection{Dancing}
Many characters have abilities that indicate that they know a specific dance. Any mechanic that requires you to ``Dance'' for something only requires both lead and follow to possess the appropriate ability card and to show it. Lead and follow are COMPLETELY UNGENDERED; anyone that knows a dance can teach it to anyone else, and any two characters who know the same dance can ``Dance'' together. If two player want to actually dance a particular dance together (it doesn't have to be the dance on the ability card; you can dance whatever the two of you want!), let the GM know and we can provide music.

\subsection{Tape on the floor}
There may be tape on the floor in this game. It represents obvious, visual effects, or an obstruction that you cannot cross. Stop and look around for a sign that explains whether you can cross this line or not.

\subsection{The Banquet}
There will be a ``banquet'' 1.5 hours into game (at the half way mark). All guests are expected to attend. There won't be any actual food at the game. Please eat breakfast before arriving to game. \cPrincess{\Prince} \cPrincess{} is organizing an event to precede the banquet.

\subsection{Magical Effects}
Magical effects in game are represented as item cards labeled ``magical effects.'' These items cannot be revealed with a normal search and are considered {\bf non-transferable} unless you know otherwise.

\subsection{NPCs}
The Neptune Ball has many NPC pages running around. When they are wearing blue headbands, they represent little minnows who serve as pages in the castle. They can carry a simple message for you to another character to the effect of: ``So-and-so wants to talk with you. I saw them in X location last.'' They may not be terribly reliable or timely however. Pages cannot carry items unless you know otherwise. NPCs will also spread game-wide announcements, and may play certain additional NPCs as necessary for some mechanics.  {\bf Pages cannot be compelled to leave a space for IC reasons, neither can pages be attacked or killed.}

\subsection{Stickers}
Placing stickers on another player represents a sketchy action like pickpocketing. If you see someone placing a sticker, you should probably ask what you see. {\bf Stickers already in place are out-of-game information.} If you have concerns about this mechanic (i.e.: You do not feel comfortable having folks trying to stick stickers on you,) please let a GM know ahead of time and we will make alternate arrangements.

\subsection{MASKS and Covid-19 Safety}
This game requires all PCs to wear masks for the duration of the event. Please put your mask on before entering the room, otherwise we'll remind you. If you do not have a mask, you will be asked to leave until you can acquire one. Players are \textbf{{highly} encouraged to wear KN95, or N95 masks for this event. If you don't have access to such masks, but want one, let us know ahead of time and we will include one in your character packet. Fabric masks can be worn \textbf{over} these more protective masks for the aesthetics if you want. With everyone in masks and HEPA filters running, understanding each other is going to be more challenging than without. Please try to speak slowly and clearly, and don't begrudge your fellow players if they ask you to repeat yourself. 

Please stay hydrated and take care of your physical wellbeing. We respectfully ask that you step outside of the play space when you need to drink water, or take your mask off temporarily for another reason.

\subsection{Other}

There is no ranged combat in this game.

There are many headbands in this game, some colors are known to everyone, some are not.
If you do not know what a headband represents, ask.

\section{Closing Notes}

These rules are imperfect.  The GMs may violate the letter of the rules to preserve the spirit.  We hope these rules are reasonably clear, but if you have any doubts about your interpretation, talk it over with us in advance.  We should also add, as much as we hate to admit it, we GMs are human: when all of our carefully laid plans are going haywire, we may not make the optimal call, and we may all have to roll with things and be a little flexible.

We hope you have lots of fun.  Good luck.

\end{document}
