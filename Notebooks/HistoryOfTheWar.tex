%%%%%
%%
%% Research Notebooks live in this directory.  This file doubles as a
%% latex'able example notebook.
%%
%% Notebook macros (in ../Lists/notebook-LIST.tex, presumably) each
%% have a file that lives here.  The argument to \startnotebook{...}
%% probably should be the macro for the given whitesheet.  However,
%% you can also just use \name{Some Text} if you want.
%%
%% Note that every \startnotebook command needs a matching
%% \endnotebook command.  Also note that no ownership information
%% appears on the notebook.
%%
%%%%%

\documentclass[greennotebook]{NeptuneBall}
\begin{document}

\startnotebook{\nWar{}}

\begin{page}{first}

As a faithful member of the Assassin's Guild, you have a mission here. The Guild is concerned that \cPrince{} is going soft, and may capitulate to many \pAtlantis{}n demands during the treaty negotiations. In order to prevent this, you have been charged with the task of discovering the true origins of the war. Once discovered, you should use them to harden \cPrince{} against \pAtlantis{}. At the start of game, turn to page \nbref{second}.

\end{page}

\begin{page}{second}

Despite the Guild's promise that all of their resources were at your disposal, you aren't entirely sure where to start. You suspect the \pAtlantis{}n people have been told nothing but lies about \pPacifica{} starting the war. Go do some research in the library and see if the history books all say the same thing. Spend 1 minute searching each bookshelf in the library. Once you have searched all 3 bookshelves, turn to page \nbref{third}.

\end{page}

\begin{page}{third}

Okay, so the lies go back to the start of the war in the books. But the facts just don't add up. How could the \pAtlantis{}ns believe it? Go and talk to \cManta{}, the oldest member of court, and see what \cManta{\they} remembers about the beginnings of the war. Give \cManta{\them} your ``\mWPacket{\MYname}'' and tell \cManta{\them} to open it. If you do not wish to speak with \cManta{}, you may go through \cManta{\their} papers in \cManta{\their} room instead. Once you have done one or the other, turn to page \nbref{fourth}.

\end{page}

\begin{page}{fourth}

\cManta{} (or \cManta{\their} papers) mentioned a \cMother{\mer} by the name of \cMother{}, but didn't give you much else of use. Yet, the name rings a bell. \cMother{} was a \pPacifica{}n \cMother{\prince}. Interesting. You're beginning to remember something you heard long ago from an old storyteller - \cMother{} definitely had something to do with the start of the war. The pieces are starting to come together for you - but you'll need a rare \pPacifica{}n text to prove it to \cPrince{}.

Luckily the Assassin's Guild has contacts in the city beyond the palace that almost certainly have a copy of the scroll you need. Hand write a request for ``\iScroll{}'' (marking it as an in game item), and give it to the nearest page (NPC) to deliver to your contact. Keep the contents of the note a secret. In 20 minutes, you may go inquire with a page to receive the scroll. Once you have it, open ``\iScroll{}'' and turn to page \nbref{fifth}.

\end{page}

\begin{page}{fifth}

This is it. This innocuous seeming scroll is a letter from \cMotherBrother{\prince} \cMotherBrother{}, to \cMotherBrother{\their} \cMother{\sibling}, \cMother{\prince} \cMother{}. It expresses \cMotherBrother{\their} deep regret that their rescue could not have come early enough to save \cMother{} from conceiving \cExExKing{}'s child. You now have proof that \pAtlantis{} started the war by kidnapping \cMother{\prince} \cMother{}. Use this information to prevent any treaty from being signed.

\end{page}

\endnotebook

\end{document}
