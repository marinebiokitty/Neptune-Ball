
%%%%%
%%
%% Research Notebooks live in this directory.  This file doubles as a
%% latex'able example notebook.
%%
%% Notebook macros (in ../Lists/notebook-LIST.tex, presumably) each
%% have a file that lives here.  The argument to \startnotebook{...}
%% probably should be the macro for the given whitesheet.  However,
%% you can also just use \name{Some Text} if you want.
%%
%% Note that every \startnotebook command needs a matching
%% \endnotebook command.  Also note that no ownership information
%% appears on the notebook.
%%
%%%%%

\documentclass[greennotebook]{NeptuneBall}
\begin{document}

%\item 
%\item 

\startnotebook{\nMusicBox{}}

\begin{page}{first}

You are devastated to find that the music box is missing. You'll have to find it tonight. First you need to talk to someone who remembers the box and understands what an important magical item it was. Find \cManta{} and discuss the music box with \cManta{\them} for at least 2 minutes. Once you have done so, you may turn to page \nbref{second}.

\end{page}

\begin{page}{second}

Well, \cManta{} couldn't give you much in the way of specifics, but your talk did do one piece of good. It has reminded you that there were inscriptions on the box. If only you could remember what those inscriptions were. Wait, of course! There must be documentation of the music box in the royal archives. You will just need to do a little research. Spend 1 minute searching each bookshelf in the library. Once you have searched all 3 bookshelves, you may turn to page \nbref{third}.

\end{page}

\begin{page}{third}

You finally found a copy of the inscription in a dusty, forgotten tome. It took some doing to decipher, but you figured out that the inscription is actually the lyrics to the tune that the music box plays! It is written in an ancient language that you happen to have been forced to study in your training as a \cAriel{\prince} of \pAtlantis{}. What luck! Unfortunately, the lyrics are not enough. Go find 3 people to help you, and 4 musical  instruments (1 for each person). Spend 2 minutes trying to remember the tune by playing together (roleplay accordingly). Once you have done so, you may turn to page \nbref{fourth}.

\end{page}

\begin{page}{fourth}

That's it! You've got it! How could you have ever forgotten that beautiful melody for a second? As the last notes of the song fade, you hear another melody. It is faint, and ebbs and flows like the tide, but it is there. The music box! It's still playing - faintly, but it's there! You may now interact with the signs on the walls and floors that have musical notes on them. Go to one of them and flip it \emph{up}. The arrows will guide your way as you follow the faint tones of music to the location of the music box. Once you ``\sPacketA{}'', turn to page \nbref{fifth}.

\end{page}

\begin{page}{fifth}

You've found it! Your \cAthena{\parent}'s music box! All you have to do now is put it back on display. Go to the ``\sArtifactZero{}'' sign, and swap it for the sign underneath. Put the music box in the envelope attached to this sign\ldots{} Something isn't quite right. The box isn't playing music any more. Oh! You need to wind it! But where is the key? Your \cAthena{\parent} must have taken it with \cAthena{\them} when \cAthena{\they} left. Where could it be now? If you manage to use \iMusicBoxKey{} to wind the \iMusicBox{}, turn to page \nbref{sixth}

\end{page}

\begin{page}{sixth}

It's done. With the music box wound, \pAtlantis{} will be safe for another decade!

\end{page}

\endnotebook

\end{document}
